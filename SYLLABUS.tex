% Education Playground - Course Syllabus
% Comprehensive Programming, AI/ML, and Computer Science Education

\documentclass[11pt,letterpaper]{article}

% Packages
\usepackage[utf8]{inputenc}
\usepackage[margin=1in]{geometry}
\usepackage{hyperref}
\usepackage{enumitem}
\usepackage{xcolor}
\usepackage{fancyhdr}
\usepackage{titlesec}
\usepackage{graphicx}
\usepackage{multicol}
\usepackage{array}

% Colors
\definecolor{primarycolor}{RGB}{0, 102, 204}
\definecolor{secondarycolor}{RGB}{51, 51, 51}

% Header/Footer
\pagestyle{fancy}
\fancyhf{}
\fancyhead[L]{\textbf{Education Playground}}
\fancyhead[R]{Course Syllabus}
\fancyfoot[C]{\thepage}

% Title formatting
\titleformat{\section}
  {\color{primarycolor}\Large\bfseries}
  {\thesection}{1em}{}[\titlerule]

\titleformat{\subsection}
  {\color{secondarycolor}\large\bfseries}
  {\thesubsection}{1em}{}

% Hyperlink setup
\hypersetup{
    colorlinks=true,
    linkcolor=primarycolor,
    urlcolor=blue,
    pdftitle={Education Playground Syllabus},
    pdfauthor={Education Playground}
}

\begin{document}

% Title Page
\begin{titlepage}
    \centering
    \vspace*{2cm}

    {\Huge\bfseries Education Playground\par}
    \vspace{0.5cm}
    {\Large Interactive Programming, AI \& Computing Platform\par}
    \vspace{2cm}

    {\LARGE\textbf{Course Syllabus}\par}
    \vspace{0.5cm}
    {\large Self-Paced Learning Curriculum\par}
    \vspace{3cm}

    \begin{tabular}{rl}
        \textbf{Version:} & 2.0 \\
        \textbf{Last Updated:} & October 2025 \\
        \textbf{Format:} & Self-Paced, Interactive Notebooks \\
        \textbf{Repository:} & \url{https://github.com/mykolas-perevicius/Education\_Playground} \\
    \end{tabular}

    \vfill
    {\large From Complete Beginner to Advanced Practitioner\par}
\end{titlepage}

% Table of Contents
\tableofcontents
\newpage

% Course Overview
\section{Course Overview}

\subsection{Mission Statement}
Education Playground is a comprehensive, self-paced learning platform designed to take students from complete beginners to advanced practitioners in programming, artificial intelligence, machine learning, and computer science fundamentals.

\subsection{Learning Philosophy}
\begin{itemize}[leftmargin=*]
    \item \textbf{Learn by Doing}: Interactive Jupyter notebooks with hands-on exercises
    \item \textbf{Flexible Paths}: Parallel tracks allow students to focus on interests
    \item \textbf{Progressive Difficulty}: Three carefully designed levels (Easy, Medium, Hard)
    \item \textbf{Real-World Focus}: Practical applications and industry-relevant skills
    \item \textbf{Complete Coverage}: From ``Hello World'' to FAANG interviews and CTF competitions
\end{itemize}

\subsection{Target Audience}
\begin{itemize}[leftmargin=*]
    \item \textbf{Beginners}: No prior programming experience required
    \item \textbf{Career Changers}: Transitioning into tech careers
    \item \textbf{Students}: Supplementing formal CS education
    \item \textbf{Professionals}: Upskilling in AI/ML and cybersecurity
    \item \textbf{Interview Prep}: Preparing for technical interviews at top companies
\end{itemize}

\subsection{Learning Outcomes}
Upon completion, students will be able to:
\begin{enumerate}[leftmargin=*]
    \item Write clean, efficient Python code following modern best practices
    \item Implement machine learning algorithms from scratch and using industry frameworks
    \item Solve classic computer science problems and algorithm challenges
    \item Understand and apply cryptographic principles
    \item Perform security analysis and ethical hacking (CTF skills)
    \item Use professional developer tools (shell, git, debugging, CI/CD)
    \item Build production-ready applications and systems
    \item Pass technical interviews at FAANG companies
\end{enumerate}

% Prerequisites
\section{Prerequisites}

\subsection{Required}
\begin{itemize}[leftmargin=*]
    \item Computer with internet connection (Windows, macOS, or Linux)
    \item Python 3.10 or higher
    \item Jupyter Notebook or JupyterLab
    \item 4GB+ RAM (8GB+ recommended for deep learning)
    \item 5GB+ free disk space
\end{itemize}

\subsection{Recommended}
\begin{itemize}[leftmargin=*]
    \item Basic computer literacy (file management, web browsing)
    \item Willingness to experiment and learn from mistakes
    \item GitHub account for version control practice
    \item Text editor (VS Code recommended)
\end{itemize}

\subsection{Optional}
\begin{itemize}[leftmargin=*]
    \item GPU with CUDA support (for deep learning acceleration)
    \item Multiple monitors (for better workflow)
    \item Mechanical keyboard (for comfort during long coding sessions)
\end{itemize}

% Course Structure
\section{Course Structure}

\subsection{Difficulty Levels}
The curriculum is organized into three difficulty levels, determined by an initial calibration test:

\begin{center}
\begin{tabular}{|c|l|l|}
\hline
\textbf{Level} & \textbf{Score Range} & \textbf{Description} \\
\hline
Easy & 0--30 points & Complete beginners, new to programming \\
Medium & 31--60 points & Basic programming knowledge \\
Hard & 61+ points & Experienced programmers \\
\hline
\end{tabular}
\end{center}

\subsection{Parallel Learning Tracks}
Each level contains three parallel tracks that can be completed in any order:
\begin{enumerate}[leftmargin=*]
    \item \textbf{Python Programming Track}: Core language skills
    \item \textbf{AI \& Machine Learning Track}: From basics to deep learning
    \item \textbf{Computing Fundamentals Track}: Computer architecture and theory
\end{enumerate}

Additionally, the \textbf{Developer Tools Track} is available at all levels.

% Easy Level
\section{Easy Level (Beginner)}

\subsection{Target Students}
Students with little to no programming experience (0--30 points on calibration test).

\subsection{Estimated Time}
20--30 hours to complete all tracks.

\subsection{Python Fundamentals Track}

\subsubsection{Lesson 1: Introduction to Python}
\begin{itemize}[leftmargin=*]
    \item Hello World and basic output
    \item Comments and documentation
    \item Running Python code
    \item ASCII art and creative exercises
\end{itemize}

\subsubsection{Lesson 2: Variables and Data Types}
\begin{itemize}[leftmargin=*]
    \item Creating and using variables
    \item Strings, integers, floats, and booleans
    \item Type checking and conversion
    \item String operations and formatting
\end{itemize}

\subsubsection{Lesson 3: Basic Operations and Conditionals}
\begin{itemize}[leftmargin=*]
    \item Mathematical operations
    \item Comparison operators
    \item If/else statements
    \item Logical operators (and, or, not)
\end{itemize}

\subsection{AI \& ML Introduction Track}

\subsubsection{Lesson 4: Introduction to AI and Machine Learning}
\begin{itemize}[leftmargin=*]
    \item What is AI and ML?
    \item Simple pattern recognition
    \item Rule-based AI systems
    \item Basic recommendation systems
\end{itemize}

\subsection{Computing Fundamentals Track}

\subsubsection{Lesson 5: Computing Fundamentals}
\begin{itemize}[leftmargin=*]
    \item How computers store information
    \item Binary numbers and data representation
    \item ASCII and text encoding
    \item Bits, bytes, and storage units
\end{itemize}

\subsection{Projects}
\begin{itemize}[leftmargin=*]
    \item Calculator application
    \item Text-based games (Hangman, Guess the Number)
    \item Simple chatbot
    \item Mad Libs generator
    \item Basic cipher encoder/decoder
\end{itemize}

% Medium Level
\section{Medium Level (Intermediate)}

\subsection{Target Students}
Students with basic programming knowledge (31--60 points on calibration test).

\subsection{Estimated Time}
40--60 hours to complete all tracks.

\subsection{Python Advanced Track}

\subsubsection{Lesson 1: Functions and Modules}
\begin{itemize}[leftmargin=*]
    \item Defining and calling functions
    \item Parameters and return values
    \item Importing and using modules
    \item Scope and namespaces
\end{itemize}

\subsubsection{Lesson 2: Data Structures}
\begin{itemize}[leftmargin=*]
    \item Lists and list operations
    \item Dictionaries for key-value storage
    \item Sets and tuples
    \item List comprehensions
\end{itemize}

\subsubsection{Lesson 3: Classes and Object-Oriented Programming}
\begin{itemize}[leftmargin=*]
    \item Creating classes and objects
    \item Instance variables and methods
    \item Constructors (\_\_init\_\_)
    \item Basic inheritance
\end{itemize}

\subsection{Machine Learning Track}

\subsubsection{Lesson 4: Machine Learning Basics}
\begin{itemize}[leftmargin=*]
    \item Introduction to scikit-learn
    \item Supervised vs unsupervised learning
    \item Classification and regression
    \item Model evaluation and accuracy
\end{itemize}

\subsubsection{Lesson 5: Data Analysis with Pandas}
\begin{itemize}[leftmargin=*]
    \item Working with DataFrames
    \item Data cleaning and manipulation
    \item Grouping and aggregation
    \item Basic data visualization
\end{itemize}

\subsection{Algorithms \& Problem Solving Track}

\subsubsection{Lesson 6: Algorithms and Problem Solving}
\begin{itemize}[leftmargin=*]
    \item Searching algorithms (linear, binary)
    \item Sorting algorithms (bubble, selection)
    \item Time complexity basics
    \item Problem-solving strategies
\end{itemize}

\subsection{Projects}
\begin{itemize}[leftmargin=*]
    \item Library management system
    \item Weather data analyzer
    \item Movie recommendation system
    \item Web scraper
    \item Sorting algorithm visualizer
\end{itemize}

% Hard Level
\section{Hard Level (Advanced)}

\subsection{Target Students}
Experienced programmers (61+ points on calibration test).

\subsection{Estimated Time}
80--120 hours to complete all tracks.

\subsection{Advanced Python Track}

\subsubsection{Lesson 1: Advanced Functions and Decorators}
\begin{itemize}[leftmargin=*]
    \item Higher-order functions and closures
    \item Lambda functions and functional programming
    \item Creating and using decorators
    \item Advanced decorator patterns
\end{itemize}

\subsubsection{Lesson 2: Generators and Iterators}
\begin{itemize}[leftmargin=*]
    \item Iterator protocol
    \item Generator functions and yield
    \item Generator expressions
    \item Coroutines and data pipelines
\end{itemize}

\subsubsection{Lesson 3: Algorithms and Complexity Analysis}
\begin{itemize}[leftmargin=*]
    \item Big O notation
    \item Sorting algorithms (quicksort, merge sort)
    \item Searching algorithms
    \item Dynamic programming
    \item Data structures (heaps, graphs, trees)
\end{itemize}

\subsection{Deep Learning \& AI Track}

\subsubsection{Lesson 4: Deep Learning and Neural Networks}
\begin{itemize}[leftmargin=*]
    \item Neural network architecture
    \item Building networks with TensorFlow/Keras
    \item Convolutional Neural Networks (CNNs)
    \item Transfer learning
    \item Advanced optimization techniques
\end{itemize}

\subsubsection{Lesson 5: Advanced ML and Natural Language Processing}
\begin{itemize}[leftmargin=*]
    \item Ensemble learning (Random Forest, XGBoost)
    \item Natural Language Processing
    \item Text classification and sentiment analysis
    \item Word embeddings
    \item Hyperparameter tuning
\end{itemize}

\subsection{Computer Systems Track}

\subsubsection{Lesson 6: Computer Systems and Theory}
\begin{itemize}[leftmargin=*]
    \item Computer architecture and organization
    \item Memory hierarchy and caching
    \item Concurrency and parallelism
    \item Computational complexity theory (P vs NP)
    \item Formal languages and automata
    \item Virtual memory and paging
\end{itemize}

\subsection{Classic Problems \& Interview Prep Track}

\subsubsection{Lesson 8: Classic Problems Collection}
\textbf{LeetCode Classics:}
\begin{itemize}[leftmargin=*]
    \item Two Sum (Hash map pattern)
    \item Three Sum (Two pointers)
    \item Longest Substring Without Repeating Characters
    \item Merge K Sorted Lists (Heap)
    \item Trapping Rain Water
\end{itemize}

\textbf{Dynamic Programming:}
\begin{itemize}[leftmargin=*]
    \item 0/1 Knapsack Problem
    \item Longest Common Subsequence (LCS)
    \item Edit Distance (Levenshtein)
\end{itemize}

\textbf{Graph Algorithms:}
\begin{itemize}[leftmargin=*]
    \item Dijkstra's Shortest Path
    \item Cycle Detection in Directed Graphs
\end{itemize}

\textbf{Cryptography:}
\begin{itemize}[leftmargin=*]
    \item Caesar Cipher
    \item Vigenère Cipher
    \item RSA Encryption Implementation
\end{itemize}

\textbf{Machine Learning from Scratch:}
\begin{itemize}[leftmargin=*]
    \item K-Nearest Neighbors
    \item Linear Regression with Gradient Descent
    \item Neural Network with Backpropagation
\end{itemize}

\subsubsection{Lesson 9: CTF Challenges - Hacker Training}
\textbf{Web Exploitation:}
\begin{itemize}[leftmargin=*]
    \item SQL Injection (OWASP Top 10)
    \item Cross-Site Scripting (XSS)
    \item Web vulnerability analysis
\end{itemize}

\textbf{Binary Exploitation (Pwn):}
\begin{itemize}[leftmargin=*]
    \item Buffer Overflow fundamentals
    \item Format String attacks
    \item Modern protections (Canaries, DEP, ASLR)
\end{itemize}

\textbf{Reverse Engineering:}
\begin{itemize}[leftmargin=*]
    \item Bytecode decompilation
    \item Crackme challenges
    \item Analysis tools (Ghidra, IDA, radare2)
\end{itemize}

\textbf{Cryptography in Practice:}
\begin{itemize}[leftmargin=*]
    \item Breaking classical ciphers
    \item Cryptanalysis techniques
\end{itemize}

\textbf{Digital Forensics:}
\begin{itemize}[leftmargin=*]
    \item LSB Steganography
    \item File Carving (magic bytes)
    \item Metadata extraction (EXIF)
\end{itemize}

\textbf{OSINT (Open Source Intelligence):}
\begin{itemize}[leftmargin=*]
    \item Information gathering techniques
    \item Privacy analysis
\end{itemize}

\textbf{Misc Challenges:}
\begin{itemize}[leftmargin=*]
    \item Multi-layer encoding/decoding
    \item Pattern recognition
\end{itemize}

\subsection{Projects}
\begin{itemize}[leftmargin=*]
    \item Custom ML framework
    \item Interpreted programming language
    \item Distributed system with consensus
    \item Blockchain implementation
    \item Operating system kernel module
    \item Compiler for subset of language
\end{itemize}

% Developer Tools Track
\section{Developer Tools Track (The Missing Semester)}

\subsection{Overview}
Essential professional development skills inspired by MIT's ``The Missing Semester of Your CS Education.'' Available at all levels.

\subsection{Estimated Time}
30--40 hours to complete all lessons.

\subsection{Core Tools Curriculum}

\subsubsection{Lesson 1: Shell and Command Line}
\begin{itemize}[leftmargin=*]
    \item Terminal fundamentals and navigation
    \item File operations and text processing
    \item Pipes, redirection, and command chaining
    \item Keyboard shortcuts and efficiency
\end{itemize}

\subsubsection{Lesson 2: Shell Scripting}
\begin{itemize}[leftmargin=*]
    \item Writing bash scripts
    \item Variables and control flow
    \item Text processing (sed, awk, grep)
    \item Automation techniques
\end{itemize}

\subsubsection{Lesson 3: Version Control with Git}
\begin{itemize}[leftmargin=*]
    \item Git fundamentals and workflows
    \item Branching, merging, and collaboration
    \item Resolving conflicts
    \item Advanced Git (rebase, cherry-pick, bisect)
\end{itemize}

\subsubsection{Lesson 4: Text Editors}
\begin{itemize}[leftmargin=*]
    \item Vim basics and philosophy
    \item Essential Vim commands
    \item VS Code power features
    \item Editor configuration
\end{itemize}

\subsubsection{Lesson 5: Data Wrangling}
\begin{itemize}[leftmargin=*]
    \item Processing text with Unix tools
    \item Regular expressions
    \item JSON/CSV manipulation
    \item Combining commands with pipes
\end{itemize}

\subsubsection{Lesson 6: Debugging and Profiling}
\begin{itemize}[leftmargin=*]
    \item Python debugger (pdb)
    \item Logging best practices
    \item Performance profiling
    \item Memory analysis
    \item Static analysis tools
\end{itemize}

\subsubsection{Lesson 7: Security Essentials}
\begin{itemize}[leftmargin=*]
    \item SSH and key-based authentication
    \item Password management
    \item Environment variables and secrets
    \item Common vulnerabilities
    \item HTTPS and certificates
\end{itemize}

\subsubsection{Lesson 8: Build Systems and CI/CD}
\begin{itemize}[leftmargin=*]
    \item Make and Makefiles
    \item GitHub Actions
    \item Testing automation
    \item Docker basics
    \item Deployment pipelines
\end{itemize}

\subsubsection{Lesson 9: Package Management}
\begin{itemize}[leftmargin=*]
    \item pip and virtual environments
    \item poetry and modern Python tools
    \item System package managers
    \item Dependency management
\end{itemize}

\subsubsection{Lesson 10: Dotfiles and Configuration}
\begin{itemize}[leftmargin=*]
    \item Shell configuration (.bashrc, .zshrc)
    \item Git configuration
    \item Tool configuration
    \item Dotfile management with Git
\end{itemize}

% Assessment and Grading
\section{Assessment and Evaluation}

\subsection{Calibration Test}
\begin{itemize}[leftmargin=*]
    \item \textbf{Purpose}: Determine appropriate starting level
    \item \textbf{Format}: Jupyter notebook with coding challenges
    \item \textbf{Duration}: Self-paced (typically 30--60 minutes)
    \item \textbf{Sections}: Basic syntax, control flow, data structures, OOP, advanced concepts
    \item \textbf{Scoring}: Self-assessment (0--100 points)
\end{itemize}

\subsection{Exercise Completion}
Each lesson contains interactive exercises:
\begin{itemize}[leftmargin=*]
    \item \textbf{Easy Level}: 3--5 exercises per lesson
    \item \textbf{Medium Level}: 5--8 exercises per lesson
    \item \textbf{Hard Level}: 8--15 exercises per lesson
    \item \textbf{Self-Check Quizzes}: Knowledge verification
\end{itemize}

\subsection{Project Work}
\begin{itemize}[leftmargin=*]
    \item Build portfolio-worthy projects
    \item Apply concepts from multiple lessons
    \item Real-world problem solving
    \item Optional: Share on GitHub for community feedback
\end{itemize}

\subsection{Progress Tracking}
\textbf{Self-Assessment Checklist:}
\begin{itemize}[leftmargin=*]
    \item Can explain concepts to others
    \item Can implement solutions from scratch
    \item Can recognize patterns and apply them
    \item Can debug efficiently
    \item Can read and understand others' code
\end{itemize}

\subsection{Mastery Indicators}
\textbf{Easy Level Mastery:}
\begin{itemize}[leftmargin=*]
    \item Write simple programs independently
    \item Understand error messages
    \item Use basic data structures
\end{itemize}

\textbf{Medium Level Mastery:}
\begin{itemize}[leftmargin=*]
    \item Design and implement complete applications
    \item Apply OOP principles
    \item Work with APIs and libraries
    \item Basic algorithm analysis
\end{itemize}

\textbf{Hard Level Mastery:}
\begin{itemize}[leftmargin=*]
    \item Solve LeetCode mediums in 20--30 minutes
    \item Implement ML algorithms from scratch
    \item Analyze time/space complexity
    \item Pass FAANG technical interviews
    \item Compete in CTF competitions
\end{itemize}

% Resources and Materials
\section{Resources and Materials}

\subsection{Core Materials}
\begin{itemize}[leftmargin=*]
    \item \textbf{Interactive Notebooks}: Jupyter notebooks with executable code
    \item \textbf{Cheat Sheets}: Quick reference guides (Python, ML/AI)
    \item \textbf{Resource Guide}: Curated external learning materials
    \item \textbf{Project Ideas}: Structured project challenges with learning goals
\end{itemize}

\subsection{Supplementary Materials}
\begin{itemize}[leftmargin=*]
    \item \textbf{SETUP.md}: Comprehensive installation guide
    \item \textbf{requirements.txt}: All necessary Python packages
    \item \textbf{README.md}: Complete course navigation
    \item \textbf{Work Log}: Session tracking for continuity
\end{itemize}

\subsection{External Resources}
\begin{itemize}[leftmargin=*]
    \item Online platforms: Futurecoder, LearnPython.org, Kaggle Learn
    \item Courses: Google ML Crash Course, Fast.ai, DeepLearning.AI
    \item Books: Automate the Boring Stuff, Think Python, Deep Learning Book
    \item Practice: LeetCode, HackTheBox, OverTheWire, picoCTF
\end{itemize}

\subsection{Tools and Software}
\begin{itemize}[leftmargin=*]
    \item \textbf{Required}: Python, Jupyter, Git
    \item \textbf{Recommended}: VS Code, Docker, GitHub CLI
    \item \textbf{Optional}: PyCharm, Vim, tmux, specialized security tools
\end{itemize}

% Learning Strategies
\section{Learning Strategies and Best Practices}

\subsection{Effective Learning Techniques}
\begin{enumerate}[leftmargin=*]
    \item \textbf{Active Learning}: Type every code example, don't just read
    \item \textbf{Experimentation}: Modify code and observe changes
    \item \textbf{Spaced Repetition}: Review previous lessons regularly
    \item \textbf{Teaching Others}: Explain concepts to solidify understanding
    \item \textbf{Project-Based}: Apply skills to real-world problems
    \item \textbf{Community Engagement}: Join forums, share progress
\end{enumerate}

\subsection{Recommended Study Schedule}

\subsubsection{Part-Time (10--15 hours/week)}
\begin{itemize}[leftmargin=*]
    \item Easy Level: 2--3 weeks
    \item Medium Level: 4--6 weeks
    \item Hard Level: 8--12 weeks
    \item Total: 3--6 months to completion
\end{itemize}

\subsubsection{Full-Time (30--40 hours/week)}
\begin{itemize}[leftmargin=*]
    \item Easy Level: 1 week
    \item Medium Level: 2 weeks
    \item Hard Level: 3--4 weeks
    \item Total: 6--8 weeks to completion
\end{itemize}

\subsection{Flexible Learning Paths}
\begin{enumerate}[leftmargin=*]
    \item \textbf{Python-First}: Master programming fundamentals before AI/ML
    \item \textbf{AI-Focused}: Learn Python alongside machine learning concepts
    \item \textbf{Systems-Focused}: Emphasize computing fundamentals and algorithms
    \item \textbf{Balanced}: Rotate between tracks for well-rounded knowledge
    \item \textbf{Interview Prep}: Focus on algorithms and classic problems
    \item \textbf{Security Path}: Prioritize CTF and developer tools tracks
\end{enumerate}

\subsection{Overcoming Challenges}
\begin{itemize}[leftmargin=*]
    \item \textbf{Stuck on Problem}: Read error messages, search documentation, ask community
    \item \textbf{Feeling Overwhelmed}: Take breaks, review fundamentals, celebrate small wins
    \item \textbf{Imposter Syndrome}: Remember everyone was a beginner once
    \item \textbf{Plateaus}: Try harder problems, build projects, learn from others
\end{itemize}

% Career Preparation
\section{Career Preparation}

\subsection{Career Paths}
\textbf{Software Engineering:}
\begin{itemize}[leftmargin=*]
    \item Backend Developer
    \item Full-Stack Developer
    \item Systems Programmer
\end{itemize}

\textbf{Data Science \& AI/ML:}
\begin{itemize}[leftmargin=*]
    \item Machine Learning Engineer
    \item Data Scientist
    \item ML Research Scientist
\end{itemize}

\textbf{Cybersecurity:}
\begin{itemize}[leftmargin=*]
    \item Penetration Tester
    \item Security Researcher
    \item Bug Bounty Hunter
    \item Incident Responder
    \item Red Team Operator
\end{itemize}

\textbf{Other Opportunities:}
\begin{itemize}[leftmargin=*]
    \item Competitive Programmer
    \item Technical Interviewer
    \item Developer Advocate
    \item Technical Writer
\end{itemize}

\subsection{Interview Preparation}
\textbf{Technical Interview Skills:}
\begin{itemize}[leftmargin=*]
    \item Pattern recognition (hash maps, two pointers, DP)
    \item Problem-solving communication
    \item Time/space complexity analysis
    \item Testing and edge cases
    \item Code optimization
\end{itemize}

\textbf{Recommended Practice:}
\begin{itemize}[leftmargin=*]
    \item LeetCode: 150+ problems (Easy: 50, Medium: 75, Hard: 25)
    \item Mock interviews with peers
    \item System design practice
    \item Behavioral question preparation
\end{itemize}

\subsection{Portfolio Building}
\begin{itemize}[leftmargin=*]
    \item Complete 3--5 substantial projects
    \item Contribute to open source
    \item Write technical blog posts
    \item Build public GitHub profile
    \item Participate in CTF competitions (showcase writeups)
\end{itemize}

% Support and Community
\section{Support and Community}

\subsection{Getting Help}
\begin{itemize}[leftmargin=*]
    \item \textbf{GitHub Issues}: Report bugs or request clarifications
    \item \textbf{Discussions}: Ask questions, share solutions
    \item \textbf{External Communities}: r/learnpython, Python Discord, Stack Overflow
\end{itemize}

\subsection{Contributing}
We welcome contributions!
\begin{itemize}[leftmargin=*]
    \item Report typos and errors
    \item Suggest improvements
    \item Add new exercises or projects
    \item Share learning resources
    \item Write tutorials or explanations
\end{itemize}

\subsection{Code of Conduct}
\begin{itemize}[leftmargin=*]
    \item Be respectful and inclusive
    \item Help others learn
    \item Give constructive feedback
    \item Share knowledge freely
    \item Practice ethical hacking (authorized testing only!)
\end{itemize}

% Appendices
\section{Appendices}

\subsection{Appendix A: Setup Instructions}
Detailed installation instructions available in \texttt{SETUP.md}:
\begin{itemize}[leftmargin=*]
    \item Platform-specific guides (Windows, macOS, Linux)
    \item Multiple installation methods (venv, conda, Docker)
    \item GPU setup for deep learning
    \item Troubleshooting common issues
\end{itemize}

\subsection{Appendix B: Quick Reference}
\begin{itemize}[leftmargin=*]
    \item \textbf{Python Cheatsheet}: \texttt{PYTHON\_CHEATSHEET.md}
    \item \textbf{ML/AI Cheatsheet}: \texttt{ML\_AI\_CHEATSHEET.md}
    \item \textbf{Resources Guide}: \texttt{RESOURCES.md}
\end{itemize}

\subsection{Appendix C: Glossary of Terms}
\begin{description}[leftmargin=*]
    \item[Algorithm] Step-by-step procedure for solving a problem
    \item[API] Application Programming Interface
    \item[Big O] Notation for describing algorithm complexity
    \item[CTF] Capture The Flag (security competition)
    \item[DP] Dynamic Programming
    \item[FAANG] Facebook/Meta, Amazon, Apple, Netflix, Google
    \item[GPU] Graphics Processing Unit (for ML acceleration)
    \item[IDE] Integrated Development Environment
    \item[ML] Machine Learning
    \item[OOP] Object-Oriented Programming
    \item[OSINT] Open Source Intelligence
    \item[Pwn] Binary exploitation challenges
    \item[RE] Reverse Engineering
    \item[REPL] Read-Eval-Print Loop
    \item[XSS] Cross-Site Scripting (web vulnerability)
\end{description}

\subsection{Appendix D: Recommended Reading Order}
\textbf{For Complete Beginners:}
\begin{enumerate}[leftmargin=*]
    \item Take calibration test
    \item Easy Level: Python Fundamentals (Lessons 1--3)
    \item Easy Level: Computing Fundamentals (Lesson 5)
    \item Developer Tools: Shell Basics (Lesson 1)
    \item Easy Level: AI Introduction (Lesson 4)
    \item Progress to Medium Level
\end{enumerate}

\textbf{For Interview Preparation:}
\begin{enumerate}[leftmargin=*]
    \item Medium Level: Algorithms \& Problem Solving
    \item Hard Level: Algorithms and Complexity
    \item Hard Level: Classic Problems Collection
    \item Practice on LeetCode daily
    \item Mock interviews
\end{enumerate}

\textbf{For Security Career:}
\begin{enumerate}[leftmargin=*]
    \item Python fundamentals
    \item Developer Tools: Shell, Git, Security Essentials
    \item Hard Level: CTF Challenges
    \item Practice on CTF platforms
    \item Learn specialized tools
\end{enumerate}

% License and Attribution
\section{License and Attribution}

\subsection{License}
This course is released under the MIT License. See \texttt{LICENSE} file for details.

\subsection{Acknowledgments}
\begin{itemize}[leftmargin=*]
    \item Inspired by MIT's ``The Missing Semester of Your CS Education''
    \item Influenced by Stanford CS courses
    \item Community contributions and feedback
    \item Open source tools and libraries
\end{itemize}

\subsection{Author}
Created by Mykolas Perevicius and the Education Playground community.

\subsection{Version History}
\begin{itemize}[leftmargin=*]
    \item \textbf{v1.0} (Initial): Basic Python curriculum
    \item \textbf{v1.5} (Major): Added AI/ML and Computing tracks
    \item \textbf{v2.0} (Current): Added Developer Tools, Classic Problems, CTF Challenges
\end{itemize}

% Contact Information
\section{Contact and Links}

\subsection{Repository}
\url{https://github.com/mykolas-perevicius/Education_Playground}

\subsection{Issues and Support}
\url{https://github.com/mykolas-perevicius/Education_Playground/issues}

\subsection{Discussions}
\url{https://github.com/mykolas-perevicius/Education_Playground/discussions}

% Final Note
\vspace{1cm}
\begin{center}
\rule{0.5\textwidth}{0.4pt}

\vspace{0.5cm}
{\large\textbf{Happy Learning!}}

\vspace{0.3cm}
{\itshape Every expert was once a beginner.
Your journey starts here. Keep practicing, stay curious,
and enjoy the journey!}

\vspace{0.5cm}
\rule{0.5\textwidth}{0.4pt}
\end{center}

\end{document}
